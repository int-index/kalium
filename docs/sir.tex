\documentclass [a4paper] {article}
\title { Sir: Sodium Internal Representation }
\author { Index Int }

\usepackage{indentfirst}

\begin {document}

\maketitle

\emph{Sodium} defines several internal (intermediate) representations. Thus
translation can be split into simpler steps: parsing, source conversion,
atomization, vectorization, target conversion, and rendering. The intermediate
representations between source and target conversions are called internal and
their specifications are provided in this document.

\section { The Scalar Representation }

The scalar representation comes in two flavors: non-atomic and atomic.
Translation of the former to the latter is called \emph{atomization}. The only
difference between the two is that in the atomic scalar representation
every expression is atomic, whereas in the non-atomic one some are not.

\subsection { Program Structure }

\begin {description}

\item [Program] is a set of functions. Every function is uniquely identified by
its signature, so a program that contains multiple functions with the same
signature is ill-formed.

\item [Function signature] contains the name of that function, types of its
parameters and the type of its result (return value).

\item [Function] consists of its signature, parameter list and a body. The
latter consists of a scope, a statement and an atomic expression---the result
that the function returns. Every parameter can be passed either by value or by
reference, and it is reflected in the signature.

The function defines two nested scopes:
\begin {enumerate}
\item Parameters scope introduces the names of parameters
\item Body scope introduces the names of local variables
\end {enumerate}

They are nested, meaning that the body scope in considered to be inside the
parameters scope. It has implications on the name shadowing.


\end {description}

\end {document}
