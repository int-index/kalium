\documentclass [a4paper] {article}
\title { Sir: Sodium Internal Representation }
\author { Index Int }

\usepackage{indentfirst}

\begin {document}

\maketitle

\emph{Sodium} defines several internal (intermediate) representations. Thus
translation can be split into simpler steps: parsing, source conversion,
atomization, vectorization, target conversion, and rendering. The intermediate
representations between source and target conversions are called internal and
their specifications are provided in this document.

\section { The Scalar Representation }

The scalar representation comes in two flavors: non-atomic and atomic.
Translation of the former to the latter is called \emph{atomization}. The only
difference between the two is that in the atomic scalar representation
every expression is atomic, whereas in the non-atomic one some are not.

\subsection { Program Structure }

\begin {description}

\item [Program] is a set of functions, every function is uniquely identified by
its name.

\item [Function] consists of a parameter list and a scoped body. The body
consists of a statement and an atomic expression---the result that the function
returns. Every parameter can be passed either by value or by reference, and it
is reflected in the signature.

A function defines two scopes:
\begin {enumerate}
\item Parameters scope introduces the names of parameters
\item Body scope introduces the names of local variables
\end {enumerate}

They are nested, meaning that the body scope is considered to be inside the
parameters scope. It has implications on the name shadowing.

\item [Scope] contains a set of variable declarations and an arbitary structure,
in which those variables can be accessed. Variable declarations can be stored in
arbitary form, from which an associative array can be derived, where the keys
are variable names, and values are types.

For example, a function defines a scope, where variable declarations are stored
as a parameter list, and the structure inside the scope is the body of that
function.

\end {description}

\end {document}
