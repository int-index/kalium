\documentclass [a4paper] {article}
\title { Kalium Internal Representation }
\author { Index Int }

\usepackage{indentfirst}

\begin {document}

\maketitle

\emph{Kalium} defines several internal (intermediate) representations. Thus
translation can be split into simpler steps: parsing, source conversion,
atomization, valuefication, vectorization, target conversion, and rendering. The
intermediate representations between source and target conversions are called
internal and their specifications are provided in this document.

\section { The Scalar Representation }

The scalar representation is a minimalistic formal system designed to represent
programs in procedural style. It comes in multiple flavors, defined by the
following properties (which it may or may not have):

\begin {description}

\item [Atomicity] means that all expressions in a program are atomic. Atomic
expressions are limited to literals and accessing variables, and thus their
evaluation is guaranteed to have no side effects. This is an important property
used during vectorization.

\item [Referencing] means that functions are allowed to use pass-by-reference
mechanism for some of their parameters. Aliasing is not allowed, though.

\end {description}

The scalar representation is statically typed. There are some built-in
polymorphic operations, but all user-defined functions and variables have a
concrete monomorphic type.

\subsection { Program Structure }

Below we describe the non-atomic referencing flavor of the scalar
representation. The other three (atomic referencing, non-atomic non-referencing,
atomic non-referencing) are the restricted versions of it.

\begin {description}

\item [Program] is a set of functions, every function is uniquely identified by
its name. There should be a \texttt{main} function that is the entry point of a
program.

\item [Function] is the main unit of abstraction. It represents an action
parametrized by values. This action also has a result --- the return value of
the function. Calling a function is providing it with some values, one for each
parameter, performing the action, and receiving the return value.

Thus, a function has a parameter list that declares what values can influence
the action and a body that defines this action. A parameter is basically a
variable declaration, annotated by a passing mechanism. The body consists of a
statement and an expression that defines the return value.

That is, a function consists of a parameter list, local variables, and a body.
Their precise scoping relation is described in another section. A function is
also annotated by the type of its return value.

\item [Variable declaration] is a simple pair. The first element is the name
of a variable, and the second element is its type.

\item [Function signature] describes how a function can be called. It contains
the types and passing mechanisms of parameters, and the type of the return
value. Function signature can be unambigously derived from the definition of a
function.

\item [Statement] is the main unit of execution. It represents an action to be
performed at runtime. A statement can contain expressions, variable declarations
and other statements. Inside a statement, variables can be accessed and
functions can be called. Different statements capture different execution
semantics.

\item [Statement \emph{exec}] calls a function with some specific values. It
optionally binds the return value to a variable. Thus, it consists of a name of
the function to be called, a list of arguments, and a variable binding pattern.
An argument is an expression which is evaluated in order to provide a function
with a value of its parameter. The arguments are evaluated left-to-right before
the call.

\item [Statement \emph{if}] performs one action or another, dependening on the
value of a condition. Thus, it consists of a boolean expression and two
statements. It evaluates the condition, and if its value is \texttt{true}, then
it executes the first statement, otherwise the value is \texttt{false} and the
second statement is executed.

\item [Statement \emph{for}] traverses a list of values, performing an action
for each. It consists of a variable name (an iterator), a list expression and a
statement. The following steps are performed:
\begin {enumerate}
\item Consider the whole list to be the remaining list;
\item If the remaining list is empty---stop, otherwise proceed;
\item Copy the value of the list head to the iterator;
\item Execute the statement;
\item Consider the list tail to be the remaining list;
\item Goto step 2.
\end {enumerate}
The value of the iterator is undefined after the statement \emph{for}.

\item [Statement \emph{follow}] performs one action after another. It consists
of two statements which are executed sequentially, one at a time.

\item [Statement \emph{pass}] performs no action.

\item [Statement \emph{scope}] introduces new variables for another statement.
The structure that contains those variables can be arbitrary.

\item [Expression] can be atomic or non-atomic. In the former case, it's either
variable access or a literal. In the latter case, it can also be a function
call, which is replaced by the return value during evaluation.

\item [Literal] represents a value of a certain type. It is stored in a form
that allows easy analysis and manipulation.

\end {description}

\subsection { Name System }

Functions and variables have unique names. There are two kinds of names:

\begin {description}

\item [Built-in] name has a special meaning. It may represent a built-in
entity or a user-defined entity with special semantics (such as the
\texttt{main} function).

\item [Numeric] name is a unique number that allows to unambiguously identify an
entity. The code that generates a scalar representation is responsible for
ensuring global uniqueness of all numeric names. No name shadowing is allowed.

\end {description}

\subsection { Scoping Rules }

A scope captures the scoping relation between two structures. The first one
introduces variables, and the second one can access them. Thus a set of variable
declarations should be derivable from the first structure.

\subsubsection { Scoping in a Function }

Parameter list creates a scope where a variable per parameter is introduced.
Inside there's another scope, created by local variables, that contains the
body. Their nesting order has no implications on semantics, since name shadowing
is forbidden.

\subsubsection { Scoping in a Statement }

Statement \emph{scope} creates a scope. The variables it introduces do not exist
outside the statement. The value of an introduced variable is undefined.

\end {document}
